\documentclass[11pt]{article}

%— Margins & encoding
\usepackage[T1]{fontenc}
\usepackage[utf8]{inputenc}
\usepackage[margin=1in]{geometry}
\usepackage{setspace}

%— Math, graphics, tables, citations
\usepackage{amsmath,amssymb}
\usepackage{graphicx}
\usepackage{booktabs}
\usepackage{natbib}
\usepackage[hidelinks]{hyperref}
\usepackage{enumitem}
\usepackage{microtype}

%— Spacing
\onehalfspacing

\DeclareUnicodeCharacter{2006}{\,}
\DeclareUnicodeCharacter{202F}{\,}

%— Title block
\title{Evaluating a Survivorship\textendash Bias\textendash Free 12--1 Month Momentum Strategy in the S\&P 500 (2005--2024)}
\author{%
  Darshan Sathish Kumar\thanks{Email: \href{mailto:darshansathishkumar@gmail.com}{darshansathishkumar@gmail.com}. Code and data: \url{https://github.com/dshan12/Momentum-Research.git}}%
}
\date{\today}

\begin{document}
\maketitle

\begin{abstract}
We re\textendash examine the canonical 12\textendash 1 month cross\textendash sectional momentum strategy in the S\&P~500 from 2005 to 2024 using a \emph{survivorship\textendash bias–free} constituent history, turnover\textendash based transaction costs, and multi\textendash factor regressions with Newey\textendash West errors. Each month we go long the top decile and short the bottom decile of the 12\textendash 1 momentum signal, equal\textendash weight both legs, and rebalance monthly. The strategy \textbf{does not} generate positive abnormal returns: the net annualized return is \(-2.07\%\) with a Sharpe ratio of \(-0.34\) (2\% annual risk\textendash free) and a maximum drawdown of \(-61\%\). Average one\textendash way turnover is \(27\%\) per month; at 10~bps per side the implied average monthly cost is \(0.027\%\). Factor regressions show that alpha is negative and statistically significant once the momentum factor (UMD) is included; \(R^2\) rises from \(\approx 0.28\) (FF5) to \(\approx 0.82\) (FF5\,+\,UMD). These findings suggest that naïve large\textendash cap momentum primarily reflects exposure to the well\textendash documented momentum factor and does not deliver residual alpha once realistic frictions are accounted for.
\end{abstract}

\noindent\textbf{Keywords:} momentum; cross\textendash sectional; turnover; transaction costs; factor models; survivorship bias.\\
\textbf{JEL:} G11; G12; C58.

\section{Introduction}
Momentum---the tendency for recent winners to continue outperforming losers---is one of the most robust empirical regularities in asset pricing \citep{Jegadeesh1993,carhart1997persistence,Asness2019}. Early studies document large abnormal returns from simple cross\textendash sectional strategies, but subsequent work emphasizes their fragility. Momentum portfolios have experienced sharp crashes \citep{Daniel2016}, while trading frictions erode profitability in practice \citep{NovyMarx2016}. Even commercial momentum indices have often underperformed broad benchmarks.

\paragraph{Research question.}
\emph{Can a naïve 12--1 month momentum rule, applied to the S\&P~500 with realistic frictions, still generate statistically significant abnormal returns?}

\paragraph{Contribution.}
This paper provides a transparent replication using freely available data and reproducible code. Specifically:
\begin{enumerate}[label=(\roman*), leftmargin=*]
  \item We eliminate survivorship bias by reconstructing dated index membership and masking returns outside inclusion months.
  \item We apply a turnover\textendash based transaction cost model calibrated at 10 basis points per side, and report sensitivity to alternative cost assumptions.
  \item We assess abnormal performance using multi\textendash factor regressions (CAPM, FF3, FF5, and FF5\,+\,UMD) with Newey--West robust errors.
\end{enumerate}

\paragraph{Preview of results.}
The strategy underperforms: net returns are negative, drawdowns severe, and alphas vanish once momentum exposure is controlled for. Our findings reinforce the interpretation of momentum as a priced risk factor rather than a persistent source of residual alpha in U.S.\ large caps.

\section{Data} \label{sec:data}
\paragraph{Universe and sample.}
The universe comprises all firms that were S\&P~500 constituents at any point between January~2005 and December~2024 (\(T{=}252\) months). Historical additions and deletions are scraped from Wikipedia and merged with Yahoo Finance adjusted\textendash close prices. Returns are computed only when a ticker is an active constituent, removing survivorship bias.

\paragraph{Price source and cleaning.}
End\textendash of\textendash month adjusted closes are downloaded via \texttt{yfinance} (\texttt{interval="1mo"}). Nonpositive prices are excluded. Monthly returns are winsorized cross\textendash sectionally at the 1st and 99th percentiles when at least 50 names are present, after filtering out obvious glitches (\(|\log(1+r)|>1.5\)).

\paragraph{Factors and benchmarks.}
Monthly Fama--French factors (Mkt\texttt{-}RF, SMB, HML, RMW, CMA), the risk\textendash free rate, and the momentum factor (UMD) are obtained from the Ken French data library. The U.S.\ market benchmark is defined as \( \text{Mkt\texttt{-}RF} + R_f \).

\paragraph{Descriptive statistics.}
Table~\ref{tab:summary} reports descriptive statistics for the strategy and the U.S.\ market. The strategy has negative mean returns, high kurtosis, and deeper drawdowns.

\begin{table}[!ht]
\centering
\caption{Summary statistics (monthly, decimals). Sample: 2005--2024, \(N{=}252\) months.}
\label{tab:summary}
\begin{tabular}{lrrrrrrrrr}
\toprule
Series & $N$ & Mean (m) & Vol (m) & Sharpe & Ann Ret & Ann Vol & Skew & Kurtosis & Max DD \\
\midrule
Strategy (net) & 252 & -0.0017 & 0.0341 & -0.34 & -0.0207 & 0.1183 & -2.283 & 16.967 & -0.613 \\
US Market      & 252 & \phantom{-}0.0094 & 0.0439 & \phantom{-}0.61 & \phantom{-}0.1184 & 0.1519 & -0.553 & \phantom{-}1.403 & -0.503 \\
\bottomrule
\end{tabular}
\end{table}

\section{Methodology} \label{sec:method}
\paragraph{Signal and portfolio formation.}
For each month \(t\), compute the cumulative return from \(t{-}13\) to \(t{-}1\). Rank stocks into percentiles; go long the top 10\% and short the bottom 10\%, equal\textendash weighted within legs. Portfolios are rebalanced monthly.

\paragraph{Transaction costs.}
Let \(w_t\) denote portfolio weights at \(t\). Dollar turnover is \(\tfrac{1}{2}\sum_i |w_{i,t+1}^{\text{pre}} - w_{i,t}|\). Net return is
\[
r^{\text{net}}_{t+1} \;=\; w_t^\top r_{t+1} \;-\; c \cdot \text{turnover}_{t+1},
\]
with \(c{=}10\)~bps. Average turnover is \(27\%\), implying costs of \(0.027\%\) per month.

\paragraph{Performance metrics.}
We report annualized mean, volatility, Sharpe (2\% RF), and maximum drawdown. Risk\textendash adjusted alphas are estimated with OLS and Newey--West standard errors (6 lags).

\section{Results} \label{sec:results}
\subsection{Performance with turnover costs}
At 10~bps per side, the strategy yields a net annualized return of \(-2.07\%\), volatility of \(11.83\%\), Sharpe of \(-0.34\), and a maximum drawdown of \(-61\%\). Performance deteriorates monotonically as costs increase (Table~\ref{tab:tc}).

\begin{table}[!ht]
\centering
\caption{Transaction\textendash cost sensitivity. Average monthly turnover $\approx 27\%$.}
\label{tab:tc}
\begin{tabular}{rrrrrr}
\toprule
Cost (bps) & Ann Ret & Vol & Sharpe & Max DD & Avg Turnover \\
\midrule
\phantom{0}5  & -1.91\% & 11.83\% & -0.33 & -60.20\% & 27.0\% \\
10 & -2.07\% & 11.83\% & -0.34 & -61.28\% & 27.0\% \\
15 & -2.23\% & 11.83\% & -0.36 & -62.33\% & 27.0\% \\
25 & -2.55\% & 11.83\% & -0.39 & -64.34\% & 27.0\% \\
\bottomrule
\end{tabular}
\end{table}

\subsection{Factor regressions}
Table~\ref{tab:ff} reports regressions of excess returns on CAPM, FF3, FF5, and FF5\,+\,UMD. Alphas are insignificant under CAPM/FF5, but turn negative and significant once UMD is included, with \(R^2\) rising to 0.82.

\begin{table}[!ht]
\centering
\caption{Factor regressions (net returns). HAC SEs, 6 lags. Alphas annualized.}
\label{tab:ff}
\begin{tabular}{lrrrr}
\toprule
Model & Alpha (ann) & t(Alpha) & $R^2$ & $n$ \\
\midrule
CAPM     & -0.72\% & -0.32 & 0.147 & 252 \\
FF3      & -1.65\% & -0.70 & 0.269 & 252 \\
FF5      & -1.56\% & -0.65 & 0.279 & 252 \\
FF5\,+\,UMD & -4.03\% & -3.23 & 0.823 & 252 \\
\bottomrule
\end{tabular}
\end{table}

\section{Conclusion} \label{sec:conclusion}
Using dated membership, turnover\textendash based costs, and factor regressions with robust errors, we find that a naïve 12--1 month momentum strategy in the S\&P~500 fails to deliver positive abnormal returns during 2005–2024. Returns are largely explained by momentum exposure, and realistic frictions render the implementation unprofitable.  

\paragraph{Limitations and future work.}
\begin{enumerate}[label=(\roman*), leftmargin=*]
  \item Our membership reconstruction relies on Wikipedia; more authoritative CRSP membership files may refine accuracy.  
  \item Turnover costs are modeled uniformly at 10~bps; liquidity\textendash adjusted costs may vary across stocks.  
  \item Factor regressions use standard Fama--French sets; exploring alternative factors or structural models may yield further insight.  
\end{enumerate}

\paragraph{Reproducibility note.}
All code, data, and scripts to replicate this analysis are available at \url{https://github.com/dshan12/Momentum-Research.git}. Tables and figures in this paper are auto\textendash generated.

\bibliographystyle{apalike}
\bibliography{references}

\end{document}

